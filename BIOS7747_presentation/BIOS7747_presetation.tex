% Options for packages loaded elsewhere
\PassOptionsToPackage{unicode}{hyperref}
\PassOptionsToPackage{hyphens}{url}
%
\documentclass[
  10pt,
  ignorenonframetext,
]{beamer}
\usepackage{pgfpages}
\setbeamertemplate{caption}[numbered]
\setbeamertemplate{caption label separator}{: }
\setbeamercolor{caption name}{fg=normal text.fg}
\beamertemplatenavigationsymbolsempty
% Prevent slide breaks in the middle of a paragraph
\widowpenalties 1 10000
\raggedbottom
\setbeamertemplate{part page}{
  \centering
  \begin{beamercolorbox}[sep=16pt,center]{part title}
    \usebeamerfont{part title}\insertpart\par
  \end{beamercolorbox}
}
\setbeamertemplate{section page}{
  \centering
  \begin{beamercolorbox}[sep=12pt,center]{part title}
    \usebeamerfont{section title}\insertsection\par
  \end{beamercolorbox}
}
\setbeamertemplate{subsection page}{
  \centering
  \begin{beamercolorbox}[sep=8pt,center]{part title}
    \usebeamerfont{subsection title}\insertsubsection\par
  \end{beamercolorbox}
}
\AtBeginPart{
  \frame{\partpage}
}
\AtBeginSection{
  \ifbibliography
  \else
    \frame{\sectionpage}
  \fi
}
\AtBeginSubsection{
  \frame{\subsectionpage}
}
\usepackage{amsmath,amssymb}
\usepackage{lmodern}
\usepackage{iftex}
\ifPDFTeX
  \usepackage[T1]{fontenc}
  \usepackage[utf8]{inputenc}
  \usepackage{textcomp} % provide euro and other symbols
\else % if luatex or xetex
  \usepackage{unicode-math}
  \defaultfontfeatures{Scale=MatchLowercase}
  \defaultfontfeatures[\rmfamily]{Ligatures=TeX,Scale=1}
\fi
\usetheme[]{CambridgeUS}
\usecolortheme{dolphin}
% Use upquote if available, for straight quotes in verbatim environments
\IfFileExists{upquote.sty}{\usepackage{upquote}}{}
\IfFileExists{microtype.sty}{% use microtype if available
  \usepackage[]{microtype}
  \UseMicrotypeSet[protrusion]{basicmath} % disable protrusion for tt fonts
}{}
\makeatletter
\@ifundefined{KOMAClassName}{% if non-KOMA class
  \IfFileExists{parskip.sty}{%
    \usepackage{parskip}
  }{% else
    \setlength{\parindent}{0pt}
    \setlength{\parskip}{6pt plus 2pt minus 1pt}}
}{% if KOMA class
  \KOMAoptions{parskip=half}}
\makeatother
\usepackage{xcolor}
\newif\ifbibliography
\setlength{\emergencystretch}{3em} % prevent overfull lines
\providecommand{\tightlist}{%
  \setlength{\itemsep}{0pt}\setlength{\parskip}{0pt}}
\setcounter{secnumdepth}{-\maxdimen} % remove section numbering
\AtBeginSubsection{}
\AtBeginSection{}
\ifLuaTeX
  \usepackage{selnolig}  % disable illegal ligatures
\fi
\IfFileExists{bookmark.sty}{\usepackage{bookmark}}{\usepackage{hyperref}}
\IfFileExists{xurl.sty}{\usepackage{xurl}}{} % add URL line breaks if available
\urlstyle{same} % disable monospaced font for URLs
\hypersetup{
  pdftitle={Learning Sample-Specific Models with Low-Rank Personalized Regression},
  pdfauthor={Benjamin Lengerich, Bryon Aragam, Eric P. Xing},
  hidelinks,
  pdfcreator={LaTeX via pandoc}}

\title{Learning Sample-Specific Models with Low-Rank Personalized
Regression}
\subtitle{Personalized regression enables sample-specific pan-cancer
analysis}
\author{Benjamin Lengerich, Bryon Aragam, Eric P. Xing}
\date{}
\institute{Carnegie Mellon University, University of Chicago}

\begin{document}
\frame{\titlepage}

\hypertarget{personalized-regression-with-regularization}{%
\section{Personalized Regression with
Regularization}\label{personalized-regression-with-regularization}}

\begin{frame}{Motivation}
\protect\hypertarget{motivation}{}
Adapting to heterogeneity in complex data to infer
\textbf{individual-level effects}

No need for the model be complex: simple linear and logistic regression
models will suffice

\begin{block}{A tradeoff between effect complexity and effect
personalization}
\protect\hypertarget{a-tradeoff-between-effect-complexity-and-effect-personalization}{}
\centering

Universal effect \(\leftrightarrow\) Personalized effect

\centering

Complex model \(\leftrightarrow\) Simple model

\begin{center}\includegraphics[width=1\linewidth]{presentation1} \end{center}
\end{block}
\end{frame}

\begin{frame}{Traditional model v.s. Personalized model}
\protect\hypertarget{traditional-model-v.s.-personalized-model}{}
\[
\begin{split}
\text {n samples:} & \ (X^{(i)}, Y^{(i)})\\
\text {Predictors:} & \ X^{(i)} \in \mathbb R^p\\
\text {Response:} & \ Y^{(i)} \\
\text {Traditional model:} & \ Y^{(i)} = f(X^{(i)}; \theta)\\
\text {Personalized model:} & \ Y^{(i)} = f(X^{(i)}; \theta^{(i)})
\end{split}
\]

\begin{itemize}
\tightlist
\item
  Multiple simple models for each individual are estimated jointly with
  a single objective function
\item
  Without additional constraints, this model is overparametrized
\item
  Definitely over-fitted, but still fixable
\end{itemize}
\end{frame}

\hypertarget{learning-sample-specific-models}{%
\section{Learning sample-specific
models}\label{learning-sample-specific-models}}

\begin{frame}{Learning sample-specific models}
\[
Y^{(i)} = f(X^{(i)}, \theta^{(i)}) + w^{(i)} \ \ \ \ \ \ (1)
\]

The key is to choose a solution \(\theta^{(i)}\) that simultaneously
leads to good generalization and accurate inferences about the \(i\)-th
sample.

\begin{itemize}
\tightlist
\item
  a low-rank latent representation of \(\theta^{(i)}\)
\item
  a novel regularization scheme.
\end{itemize}
\end{frame}

\begin{frame}{Low-rank representation}
\protect\hypertarget{low-rank-representation}{}
Ideas of dictionary learning: good-behaved personalized modeling has a
sparse representation

\[
\begin{aligned}
\text {Low rank:} & \ \Omega = [\theta^{(1)} | ... | \theta^{(n)}] \in \mathbb R^{p \times n}\\
 & i.e.  \ \theta^{(i)}  = Q^T Z^{(i)}\\
\text {Loadings:}  & \ Z^{(i)} \in \mathbb R^q;   Z \in \mathbb R^{q \times n}\\
\text {Dictionary:}  & \  Q \in \mathbb R^{q\times p}\\
\text {Low rank:} & \ \Omega = Q^T Z\\
\end{aligned}
\]

\begin{block}{Special form of sparse dictionary learning aka sparse
coding}
\protect\hypertarget{special-form-of-sparse-dictionary-learning-aka-sparse-coding}{}
\begin{itemize}
\tightlist
\item
  Sparse coding is a representation learning method which aims at
  finding a sparse representation of the input data
\item
  In the form of a linear combination of basic elements as well as those
  basic elements themselves
\item
  These elements are called atoms and they compose a dictionary.
\end{itemize}
\end{block}
\end{frame}

\begin{frame}{}
\protect\hypertarget{section}{}
The choice of \(q\) is determined by the user's desired latent
dimensionality;

for \(q \ll p\), using only \(\Theta (q(n+p))\) instead of
\(\Theta(np)\) of a full-rank solution can improve computational and
statistical efficiency.

With \(\theta^{(i)} = Q^T Z^{(i)}\), lower rank formulation enables to
use \(L_2\) distance in \(Z\) to restrict Euclidean distances between
the \(\theta^{(i)}\)

\[
\|\theta^{(i)} - \theta^{(j)} \| \leq \sqrt p \|Z^{(i)} - Z^{(j)}\| \ \ \ \ \ \ (2)
\]

\begin{itemize}
\item
  Sparsity in \(\theta\) can be realized by sparsity in \(Z\), \(Q\)
\item
  The low-rank formulation constrains the number of personalized
  sparsity patterns, by changing the latent dimensionality \(q\).
\end{itemize}
\end{frame}

\begin{frame}{Distance-matching}
\protect\hypertarget{distance-matching}{}
Regularize the parameters \(\theta^{(i)}\) by requiring that similarity
in \(\theta\) corresponds to similarity in \(U\),

\begin{itemize}
\item
  i.e.~\(\| \theta^{(i)} - \theta ^{(j)} \| \approx \rho (U^{(i)}, U^{(i)})\)
\item
  The \(U(i)\) represent exogenous variables that we do not wish to
  directly model.
\end{itemize}
\end{frame}

\begin{frame}{Distance-matching regularization (DMR)}
\protect\hypertarget{distance-matching-regularization-dmr}{}
Adapt a distance-matching regularization (DMR) scheme to penalize the
squared difference in implied distances.

The covariate distances are modeled as a weighted sum.

\[
\rho_{\phi} (u, v) = \sum^k_{l = 1} \phi_l d_l(u_l, v_l), \ \phi_l\geq 0  \ \ \ \ \ \ (3)
\]

\tiny

Each \(d_l (l = 1, ... , k)\) is a metric for a covariate;

\(\phi\) is a positive (non-negative) vector represents a linear
transformation into latent distance function;
\end{frame}

\begin{frame}{Distance Matching Regularized Loss}
\protect\hypertarget{distance-matching-regularized-loss}{}
Main idea: Distance between sample parameters should be similar to
distance between sample covariates, as well as the distance between
latent variable \(U\)

In order for
\(\|\theta^{(i)} - \theta^{(j)}\| \approx \rho_{\phi}(U^{(i)}, U^{(j)})\),
it is suffices to require
\(\|Z^{(i)} - Z^{(j)}\| \approx \rho_{\phi}(U^{(i)}, U^{(j)})\)

Then we can define the \textbf{distance-matching regularizer}:

\[
D^{(i)}_{\gamma} (d_{\beta}, d_U) = \frac {\gamma} {2} \sum_{j \in B_r(i)} \bigg( \rho_\phi (U^{(i)}, U^{(j)}) - \|Z^{(i)}, Z^{(j)}\|_2\bigg)^2 \ \ \ \ \ \ \ \ (4)
\]

\tiny

\(B_r(i) = \{j : \|Z^{(i)} - Z^{(j)}\|^2 < r\}\)

The hyperparameter \(\gamma\) trades off sensitivity to prediction of
the response variable against sensitivity to covariate structure

For example for simple linear relationship both under \(L_2\) norm

\[
D^{(i)}_{\gamma} (d_{\beta}, d_U) = \frac {\gamma} {2} \sum_{j \neq i} \bigg( d_\theta (\theta^{(i)}, \theta^{(j)}) - d_Z(Z^{(i)}, Z^{(j)})\bigg)^2 = \frac {\gamma} {2} \sum_{j \neq i} \bigg( \|\theta^{(i)}, \theta^{(j)}\|_2^2 - \|Z^{(i)}, Z^{(j)}\|_2^2 \bigg)^2
\]
\end{frame}

\begin{frame}{Personalized Regression}
\protect\hypertarget{personalized-regression}{}
Seeking a model for inference, not necessarily for accurate predictive
models

Seeking relatively simple personalized effects not universal effects

covariate data as informative of each sample

Here is the \(\ell (x, y, \theta)\) is the loss function we want to
minimize

\[
\mathcal L^{(i)} (Z, Q, \phi) =  \ell \big(X^{(i)}, Y^{(i)}, Q^TZ^{(i)}\big) + \psi_\lambda(Q^TZ^{(i)}) + D_\gamma^{(i)}(Z, \phi) \ \ \ \  (5)
\]

\tiny

\begin{itemize}
\item
  where \(\psi_\lambda\) is a regularization such as \(L_1\) penalty
\item
  \(D_\gamma^{(i)}\) is the distance-matching regularization defined in
  \(Eq.(4)\)
\end{itemize}
\end{frame}

\begin{frame}{}
\protect\hypertarget{section-1}{}
Where we learn \(\Omega\) and \(\phi\) by minimizing the following
objective:

\[
\pmb {\mathcal L} (Z, Q, \phi) = \sum^n_{i=1} \mathcal L^{(i)}(Z, Q, \phi) + v \|\phi - 1\|^2_2 \ \ \ \ \ \ \ \ (6)
\]

\tiny

\begin{itemize}
\item
  where \(v \|\phi - 1\|^2_2\) regularize the distance function
  \(\rho_\phi\) with strength set \(v\)
\item
  again, \(\Omega = Q^TZ\)
\end{itemize}
\end{frame}

\begin{frame}{Algorithm}
\protect\hypertarget{algorithm}{}
\begin{itemize}
\item The objective function is optimized with sub-gradient descent.
\item Initialize $\Sigma$ by setting $\theta^{(i)}$~$N(\hat\theta, \varepsilon I)$ for population model such as lasso and elastic net.
\item Initialize $Z$ and $Q$ by PCA factorization.
\item Each personalized estimator is endowed with a personalized learning rate $\alpha_t^{(i)} = \alpha_t / ||\hat \theta_t^{(i)} - \hat \theta^{(pop)}||_{\infty}$.
\item  This learning rate scales the global learning rate $\alpha_t$ according to how far the estimator has traveled.
\item This scheme ensures that the personalized coefficients' center of mass stays close to the initial $\hat \theta^{(pop)}$ despite unconstrained $\theta^{(i)}$.
\end{itemize}
\end{frame}

\begin{frame}{Algorithm}
\protect\hypertarget{algorithm-1}{}
\begin{center}\includegraphics[width=1\linewidth]{presentation6} \end{center}
\end{frame}

\begin{frame}{Prediction}
\protect\hypertarget{prediction}{}
\begin{itemize}
\item
  Given a test point \((X, U)\), averaging the model parameters of the
  \(k_n\) nearest training points based on distance \(\rho_\phi\)
\item
  Increasing \(k_n\) drives the test models toward the population model
  to control overfitting
\item
  Intentionally avoided using \(X\) to select \(\theta\) so that
  interpretation of \(\theta\) is not confounded by \(X\)
\end{itemize}

\[
\theta = \frac 1 {k_n} \sum^{k_n}_{j=1} \theta^{(\eta(\rho_\phi, U))[j]}; \ \eta(\rho_\phi, U) = {{argsort} _{a\leq i \leq n}} \rho_\phi (U, U^{(i)}) \ \ \ \ \ \ \ \ (7.a)
\]
\end{frame}

\begin{frame}{Results}
\protect\hypertarget{results}{}
\begin{itemize}
\item Performance of personalized regression will be tested through simulation study.
\item Fix $X \in \mathbb{R}^{N \times P}$, generate sample-specific $\beta^{(i)}$ ~ Unif (0,1), and $Y^{(i))} \in (0,1)$.
\item Covariates $U^{(i)}$ are generated by projecting $\beta^{(j)}$ into $K < P$ dimension with mult-dimensional scaling.
\item covariates that are related to the personalized regression coefficients in a highly nonlinear, nonparametric manner.
\item Set $K = 10, K = 3$.
\end{itemize}

\begin{center}\includegraphics[width=0.5\linewidth]{presentation11} \end{center}
\end{frame}

\begin{frame}{Simulation}
\protect\hypertarget{simulation}{}
\begin{center}\includegraphics[width=0.7\linewidth]{presentation10} \end{center}
\end{frame}

\begin{frame}{Sample-specific pan-cancer analysis}
\protect\hypertarget{sample-specific-pan-cancer-analysis}{}
They also use gene expression (RNA-Seq) quantification data from The
Cancer Genome Atlas (TCGA).

\begin{itemize}
\item
  This dataset compiles data from 37 projects spanning 36 disease types
  in 28 primary sites.
\item
  After pruning for missing values, this dataset contains 9663 profiles
  for 8944 case and 719 matched control samples
\item
  This resulting in \(P = 4123\) features when an intercept term is
  added.
\item
  They divide this set into 75\% training data and 25\% testing.
\end{itemize}
\end{frame}

\begin{frame}{PR has good variable selections}
\protect\hypertarget{pr-has-good-variable-selections}{}
\begin{center}\includegraphics[width=0.6\linewidth]{presentation9} \end{center}
\end{frame}

\begin{frame}{PR has good variable selections}
\protect\hypertarget{pr-has-good-variable-selections-1}{}
\begin{center}\includegraphics[width=0.4\linewidth]{presentation12} \end{center}
\end{frame}

\begin{frame}{Here is the result of RNA-seq data}
\protect\hypertarget{here-is-the-result-of-rna-seq-data}{}
\begin{center}\includegraphics[width=0.6\linewidth]{presentation8} \end{center}
\end{frame}

\begin{frame}{Cancer is a highly personalized problem}
\protect\hypertarget{cancer-is-a-highly-personalized-problem}{}
\begin{center}\includegraphics[width=0.7\linewidth]{presentation3} \end{center}
\end{frame}

\begin{frame}{The clusters have biological and clinical meanings}
\protect\hypertarget{the-clusters-have-biological-and-clinical-meanings}{}
\begin{center}\includegraphics[width=0.45\linewidth]{presentation5} \end{center}
\end{frame}

\begin{frame}{Discussion}
\protect\hypertarget{discussion}{}
\begin{itemize}
\tightlist
\item
  The Generalization Problem? Sensitive to outliers?
\item
  Is it really interpreting as its claimed to be? Confounding and
  Collinearity?
\item
  Tuning the parameters \(\lambda, \gamma. v, c, \alpha\)?
\item
  Why PCA? How Kernel PCA? ?
\end{itemize}
\end{frame}

\end{document}
